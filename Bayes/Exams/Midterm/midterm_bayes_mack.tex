\documentclass{article}\usepackage[]{graphicx}\usepackage[]{color}
%% maxwidth is the original width if it is less than linewidth
%% otherwise use linewidth (to make sure the graphics do not exceed the margin)
\makeatletter
\def\maxwidth{ %
  \ifdim\Gin@nat@width>\linewidth
    \linewidth
  \else
    \Gin@nat@width
  \fi
}
\makeatother

\definecolor{fgcolor}{rgb}{0.345, 0.345, 0.345}
\newcommand{\hlnum}[1]{\textcolor[rgb]{0.686,0.059,0.569}{#1}}%
\newcommand{\hlstr}[1]{\textcolor[rgb]{0.192,0.494,0.8}{#1}}%
\newcommand{\hlcom}[1]{\textcolor[rgb]{0.678,0.584,0.686}{\textit{#1}}}%
\newcommand{\hlopt}[1]{\textcolor[rgb]{0,0,0}{#1}}%
\newcommand{\hlstd}[1]{\textcolor[rgb]{0.345,0.345,0.345}{#1}}%
\newcommand{\hlkwa}[1]{\textcolor[rgb]{0.161,0.373,0.58}{\textbf{#1}}}%
\newcommand{\hlkwb}[1]{\textcolor[rgb]{0.69,0.353,0.396}{#1}}%
\newcommand{\hlkwc}[1]{\textcolor[rgb]{0.333,0.667,0.333}{#1}}%
\newcommand{\hlkwd}[1]{\textcolor[rgb]{0.737,0.353,0.396}{\textbf{#1}}}%

\usepackage{framed}
\makeatletter
\newenvironment{kframe}{%
 \def\at@end@of@kframe{}%
 \ifinner\ifhmode%
  \def\at@end@of@kframe{\end{minipage}}%
  \begin{minipage}{\columnwidth}%
 \fi\fi%
 \def\FrameCommand##1{\hskip\@totalleftmargin \hskip-\fboxsep
 \colorbox{shadecolor}{##1}\hskip-\fboxsep
     % There is no \\@totalrightmargin, so:
     \hskip-\linewidth \hskip-\@totalleftmargin \hskip\columnwidth}%
 \MakeFramed {\advance\hsize-\width
   \@totalleftmargin\z@ \linewidth\hsize
   \@setminipage}}%
 {\par\unskip\endMakeFramed%
 \at@end@of@kframe}
\makeatother

\definecolor{shadecolor}{rgb}{.97, .97, .97}
\definecolor{messagecolor}{rgb}{0, 0, 0}
\definecolor{warningcolor}{rgb}{1, 0, 1}
\definecolor{errorcolor}{rgb}{1, 0, 0}
\newenvironment{knitrout}{}{} % an empty environment to be redefined in TeX

\usepackage{alltt}

\usepackage{fancyhdr} % Required for custom headers
\usepackage{lastpage} % Required to determine the last page for the footer
\usepackage{extramarks} % Required for headers and footers
\usepackage{graphicx} % Required to insert images
\usepackage{hyperref}
\usepackage{amsmath} %for binomial pdf
\usepackage{parskip} % so that there's space bw paragraphs
\usepackage{float}
\usepackage{amsfonts}

% Margins
\topmargin=-0.45in
\evensidemargin=0in
\oddsidemargin=0in
\textwidth=6.5in
\textheight=9.0in
\headsep=0.25in 

\linespread{1.1} % Line spacing

% Set up the header and footer
\pagestyle{fancy}
\lhead{STAT 532: Bayes} % Top left header
\chead{Midterm: Take Home} % Top center header
\rhead{Andrea Mack} % Top right header
\lfoot{10/17/2016} % Bottom left footer
\cfoot{} % Bottom center footer
\rfoot{Page\ \thepage\ of\ \pageref{LastPage}} % Bottom right footer
\renewcommand\headrulewidth{0.4pt} % Size of the header rule
\renewcommand\footrulewidth{0.4pt} % Size of the footer rule

\setlength\parindent{0pt} % Removes all indentation from paragraphs
\setlength\parskip{0.5cm}
\restylefloat{table}

%----------------------------------------------------------------------------------------
%	DOCUMENT STRUCTURE COMMANDS
%	Skip this unless you know what you're doing
%----------------------------------------------------------------------------------------

% Header and footer for when a page split occurs within a problem environment
\newcommand{\enterProblemHeader}[1]{
\nobreak\extramarks{#1}{#1 continued on next page\ldots}\nobreak
\nobreak\extramarks{#1 (continued)}{#1 continued on next page\ldots}\nobreak
}

% Header and footer for when a page split occurs between problem environments
\newcommand{\exitProblemHeader}[1]{
\nobreak\extramarks{#1 (continued)}{#1 continued on next page\ldots}\nobreak
\nobreak\extramarks{#1}{}\nobreak
}


%----------------------------------------------------------------------------------------%
\IfFileExists{upquote.sty}{\usepackage{upquote}}{}
\begin{document}


\begin{enumerate}
\addtocounter{enumi}{1}
\item%2
{\bf Capital Bikers -- almost 400 hubs, over 3500 bikes -- Estimate $\theta$, the true average number of bikes rented at a hub across the system.}


\begin{enumerate}
\item%2a

The Poison model, parameterized by $\theta$, is a reasonable sampling model because we are interested in estimating a mean from count data, the mean count of bikes, which is reasonably finite, within a closed space, the Washington D.C. area.

Here we have data on 392 hubs, if the hubs can be thought of as independent, then $\Sigma y_{i}$ $\sim$ POISON($\Sigma \theta_{i}$), where $y_{i}$ represents the count of the number of bikes rented at each of the ith hubs and $\theta_{i}$ is the true number of bikes rented at each of the ith hubs.

For this problem, I will say that $\theta_{i}$ = $\theta$.

\item%2b
The parameter to estimate with a Poison model is $\theta$, which here represents the true mean number of bikes as well as the true variation in the number of bikes. Counts of bikes will only be positive, as well as the mean and the variance is always positive. The Gamma model has a support of (0,$\infy$) making it an ideal choice for the prior distribution on $\theta$. It is also ideal because it weakens the direct mean-variance relationship seen in the Poison model by introducing an extra parameter in to the pdf.

\item%2c
If $\underset{\sim}{y} | \theta$ $\sim$ POISON($\theta$), $\theta$ $\sim$ GAMMA($\alpha$,$\beta$), then $\theta | \underset{\sim}{y}$ $\sim$ GAMMA($\alpha + \Sigma y_{i}$, $\beta + n$).

$\int \frac{\theta^{\Sigma y_{i}}e^{-n\theta}}{\Sigma y_{i}!} \times \frac{\beta^{\alpha}e^{-\beta\theta}\theta^{\alpha - 1}}{\Gamma(\alpha)} d\theta$

$\propto$



\end{enumerate}
\end{enumerate}

\appendix
\section*{R Code}

\begin{enumerate}

\setcounter{enumi}{6}
\item%7
\begin{enumerate}
\item%7a

\addtocounter{enumii}{1}
\item %7c



\item %7d


\end{enumerate}
\end{enumerate}


\end{document}
